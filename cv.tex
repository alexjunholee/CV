%%%%%%%%%%%%%%%%%%%%%%%%%%%%%%%%%%%%%%%%%%%%%%%%%%%%%%%%%%%%%%%%%%%%%%%%%%%%%%%%
% Medium Length Graduate Curriculum Vitae
% LaTeX Template
% Version 1.2 (3/28/15)
%
% This template has been downloaded from:
% http://www.LaTeXTemplates.com
%
% Original author:
% Rensselaer Polytechnic Institute 
% (http://www.rpi.edu/dept/arc/training/latex/resumes/)
%
% Modified by:
% Daniel L Marks <xleafr@gmail.com> 3/28/2015
%
% Important note:
% This template requires the res.cls file to be in the same directory as the
% .tex file. The res.cls file provides the resume style used for structuring the
% document.
%
%%%%%%%%%%%%%%%%%%%%%%%%%%%%%%%%%%%%%%%%%%%%%%%%%%%%%%%%%%%%%%%%%%%%%%%%%%%%%%%%

%-------------------------------------------------------------------------------
%	PACKAGES AND OTHER DOCUMENT CONFIGURATIONS
%-------------------------------------------------------------------------------

%%%%%%%%%%%%%%%%%%%%%%%%%%%%%%%%%%%%%%%%%%%%%%%%%%%%%%%%%%%%%%%%%%%%%%%%%%%%%%%%
% You can have multiple style options the legal options ones are:
%
%   centered:	the name and address are centered at the top of the page 
%				(default)
%
%   line:		the name is the left with a horizontal line then the address to
%				the right
%
%   overlapped:	the section titles overlap the body text (default)
%
%   margin:		the section titles are to the left of the body text
%		
%   11pt:		use 11 point fonts instead of 10 point fonts
%
%   12pt:		use 12 point fonts instead of 10 point fonts
%
%%%%%%%%%%%%%%%%%%%%%%%%%%%%%%%%%%%%%%%%%%%%%%%%%%%%%%%%%%%%%%%%%%%%%%%%%%%%%%%%

\documentclass[margin]{res}  

% Default font is the helvetica postscript font
\usepackage{helvet}
\usepackage{hyperref}
\usepackage{kotex}

% Increase text height
\textheight=700pt

\begin{document}
	
	%-------------------------------------------------------------------------------
	%	NAME AND ADDRESS SECTION
	%-------------------------------------------------------------------------------
	\name{이 준 호 Alex Junho Lee}
	
	% Note that addresses can be used for other contact information:
	% -phone numbers
	% -email addresses
	% -linked-in profile
	
	\address{Senior Researcher \\ Robotics Lab. \\ Hyundai Motor Company}
	\address{\href{https://c11.kr/alexlee}{Website}\\alexjunholee@gmail.com\\ \href{https://scholar.google.com/citations?user=SM7JFHsAAAAJ}{Google Scholar}
	}
	
	% Uncomment to add a third address
	%\address{Address 3 line 1\\Address 3 line 2\\Address 3 line 3}
	%-------------------------------------------------------------------------------
	
	\begin{resume}
		
		%-------------------------------------------------------------------------------
		\section{Education}
		\textbf{Korea Advanced Institute of Science and Technology (KAIST)}\\
		{\sl Ph.D.}, Civil and Environmental Engineering, 2023
		\\- Robotics Program, Smart City Program
		\\
		{\sl Bachelor of Engineering}, Mechanical Engineering, Feb 2017
		\hfill GPA: 3.58 / 4.3
		\\- Double Major in Business and Technology Management (BTM)

		%-------------------------------------------------------------------------------
		\section{Field of Interests}
		Visual Localization, Multimodal sensor fusion, SLAM, Place Recognition, Spatial AI

		%-------------------------------------------------------------------------------
		\section{Publications}
		\par\underline{Alex Junho Lee}, Ayoung Kim, ``Event-based Real-time Optical Flow Estimation''. In Proceedings of the \textit{IEEE International Conference on Control, Automation and Systems (ICCAS)}, 2017.

		\par\underline{Alex Junho Lee}, Younggun Cho, Sungho Yoon, Joowan Kim, Ayoung Kim, ``ViViD: Vision for Visibility Dataset". In Proceedings of the \textit{IEEE International Conference on Robotics and Automation (ICRA) Workshop: Dataset Generation and Benchmarking of SLAM Algorithms for Robotics and VR/AR, Best Paper}, 2019.
		
		\par\underline{Alex Junho Lee}, Ayoung Kim, ``EventVLAD: Visual Place Recognition with Reconstructed Edges from Event Cameras". In Proceedings of the \textit{IEEE/RSJ International Conference on Intelligent Robots and Systems (IROS)}, 2021.
		
		\par\underline{Alex Junho Lee}, Hyun Myung, ``Natural Language Representation as Features for Place Recognition''. In Proceedings of the \textit{IEEE International Conference on Ubiquitous Robots (UR)}, 2022.
		
		\par\underline{Alex Junho Lee}, Younggun Cho, Young-sik Shin, Ayoung Kim, Hyun Myung, ``ViViD++ : Vision For Visibility Dataset". \textit{IEEE Robotics and Automation Letter (RA-L)}, 7(3):6282-6289, 2022.
		
		\par\underline{Alex Junho Lee}, Younggun Cho, Hyun Myung, ``Low-cost Thermal Mapping for Concrete Heat Monitoring". In Proceedings of the \textit{IEEE International Conference on Robotics and Automation (ICRA) Workshop: Future of Construction: Build Faster, Better, Safer - Together with Robots}, 2022.
		
		\par\underline{Alex Junho Lee}, Hyungtae Lim, Minho Oh, Wonho Song, Hyun Myung, ``Volumetric Vegetation Monitioring from LiDAR Scans with Ground Estimation.''. In Proceedings of the \textit{IEEE International Conference on Control, Automation and Systems (ICCAS)}, 2022
		
		\par\underline{Alex Junho Lee}, Wonho Song, Byeongho Yu, Duckyu Choi, Christian Tirtawardhana, Hyun Myung, ``Survey of Robotics Technologies for Civil Infrastructure Inspection" In \textit{Journal of Infrastructure Intelligence and Resilience (JIIR)}, 2022.
		
		\par\underline{Alex Junho Lee}, Seungwon Song, Hyungtae Lim, Woojoo Lee, Hyun Myung, ``(LC)$^2$: LiDAR-Camera Loop Constraints
		For Cross-Modal Place Recognition". \textit{IEEE Robotics and Automation Letter (RA-L)}, 2023.
		
		\newpage
		%-------------------------------------------------------------------------------
		\section{Achievements}
		
		\par Best Paper,\textit{ IEEE Int. Conf. Robotics and Automation (ICRA) Workshop: Dataset Generation and Benchmarking of SLAM Algorithms for Robotics and VR/AR}, 2019.
						
		\par Co-Chair, Localization II, \textit{IEEE/RSJ International Conference on Intelligent Robots and Systems (IROS)}, 2021.
		
		\par 2nd Cash Award, \textit{HILTI SLAM CHALLENGE}, 2022.

		%-------------------------------------------------------------------------------
		\section{Experiences}
		\par	Robust visual place recognition for location authentication (Project, 2022) 
		\\-	Deep learning-based VPR, participated as project leader.
		\par	Last-mile delivery robot in urban crowded areas (Project, 2021-2022) 
		\\-	LiDAR-based SLAM for UGV, participated as SLAM engineer.
		\par	Visual SLAM on racing drones (Final Stage, 2021)
		\\-	Korean DARPA Challenge, participated as SLAM part engineer.
		\\-	Stereo VIO and LiDAR map building on embedded device (Jetson TX2)
		\par	Future directions of robotics research (Project, 2021)
		\\-	Investigation project for robotics research plans, participated as team member.
		\par	Outdoor SLAM in unstructured environment (Project, 2019-2021)
		\\-	Autonomous map building in construction sites, participated as project leader.
		\\-	Active SLAM, Long-term mapping, Sensor Integration
		\par	Indoor SLAM with dynamic obstacles (Project, 2019)
		\\-	Indoor service robot for general uses, participated as SLAM part engineer
		\\-	SLAM in dynamic environment and obstacles, Obstacle avoidance.
		\par	Encoder frame device and vehicle odometry measurement system (Patent, 2019)
		\\-	High-resolution encoder frame for vehicle odometry, suggested and built hardware.
		\par	Indoor SLAM under complex disaster (Project, 2018)
		\\-	SLAM under environmental disturbances (Dust, Heat), participated as team member.
		\par	4th Industrial revolution and autonomous driving (Project, 2017)
		\\-	Investigation project for autonomous driving, participated as team member.
		\par	Intern in safety design department (Doosan Heavy Industries, 2016)
		\par	International student exchange program (National University of Singapore, 2016)
		\par	Teacher (KAIST Global Center for Gifted Children, 2015-2018)
		\par	Education volunteering (Daejeon Yuseong-gu, 2013-2015)
		
		%-------------------------------------------------------------------------------
		
		\section{Language}
		\par Korean (Native), English (Fluent)
		\par Python, C++, MATLAB
				
	\end{resume}
\end{document}